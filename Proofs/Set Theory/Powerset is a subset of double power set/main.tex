\documentclass{article}
\usepackage[utf8]{inputenc}
\usepackage[mathscr]{euscript}
\let\euscr\mathscr \let\mathscr\relax% just so we can load this and rsfs
\usepackage[scr]{rsfso}
\usepackage{amsmath}
\usepackage{makecell}


\author{Iaroslav Baranov (kciray8@gmail.com)}
\date{\today}
\newcommand{\powerset}{\raisebox{.15\baselineskip}{\Large\ensuremath{\wp}}}

\newcommand{\ps}{\mathscr{P}}

\def\arraystretch{1.5}%
\setlength{\parindent}{0pt}
\usepackage{geometry}
\geometry{a4paper,total={170mm,257mm},left=10mm,top=10mm, paperheight=240cm,  layoutheight=240cm, textheight=3400}
\pdfpageheight=2500mm
\begin{document}

\maketitle
\section{Source}
Book: How to Prove It: A Structured Approach 3rd Edition

ISBN-13: 978-1108439534

Exercise: 3.3.4

\section{Task}
Suppose  $A \subseteq  \mathscr{P}(A)$. Prove that $  \mathscr{P}(A) \subseteq \mathscr{P}(\mathscr{P}(A))$

\section{Scratch Work}
Assume $A \subseteq  \mathscr{P}(A)$

\begin{tabular}{ | c | c| } 
\hline
Givens & Goal \\ 
\hline
$A \subseteq  \mathscr{P}(A)$ 
& 
$  \mathscr{P}(A) \subseteq \mathscr{P}(\mathscr{P}(A))$ 
\\ 
\hline
\end{tabular}

\bigskip
Apply the definition of a subset

\begin{tabular}{|c|c|} 
\hline
Givens & Goal \\ 
\hline
$A \subseteq  \mathscr{P}(A)$ 
& 
$  \forall x ( x \in \mathscr{P}(A) \implies x \in \mathscr{P}(\mathscr{P}(A)) )$ 
\\ 
\hline
\end{tabular}

\bigskip
Let x be an arbitrary element

\begin{tabular}{|c|c|} 
\hline
Givens & Goal \\ 
\hline
$A \subseteq  \mathscr{P}(A)$ 
& 
$   x \in \mathscr{P}(A) \implies x \in \mathscr{P}(\mathscr{P}(A)) $ 
\\ 
\hline
\end{tabular}


\bigskip
Assume x \in \mathscr{P}(A)

\begin{tabular}{|c|c|} 
\hline
Givens & Goal \\ 
\hline
\makecell{
$A \subseteq  \mathscr{P}(A)$  
\\ 
x \in \mathscr{P}(A)
}
& 
$   x \in \ps(\ps(A)) $ 
\\ 
\hline
\end{tabular}
 
\bigskip
By the definition of a power set

\begin{tabular}{|c|c|} 
\hline
Givens & Goal \\ 
\hline
\makecell{
$A \subseteq  \mathscr{P}(A)$  
\\ 
x \in \mathscr{P}(A)
}
& 
$   x \subseteq \ps(A) $ 
\\ 
\hline
\end{tabular}
 
\bigskip
Apply the definition of a subset

\begin{tabular}{|c|c|} 
\hline
Givens & Goal \\ 
\hline
\makecell{
$A \subseteq  \mathscr{P}(A)$  
\\ 
x \in \mathscr{P}(A)
}
& 

$  \forall y ( y \in x \implies y \in \ps(A) )$ 

\\ 
\hline
\end{tabular}



\bigskip
Let y be an arbitrary element of x

\begin{tabular}{|c|c|} 
\hline
Givens & Goal \\ 
\hline
\makecell{
$A \subseteq  \mathscr{P}(A)$  
\\ 
x \in \mathscr{P}(A)
}
& 

$  y \in x \implies y \in \ps(A)$ 

\\ 
\hline
\end{tabular}

\bigskip
Assume y \in x
\newline
\nopagebreak
\begin{tabular}{|c|c|} \hline
Givens & Goal \\ \hline
\makecell{ %Givens
$A \subseteq  \mathscr{P}(A)$  
\\ 
x \in \mathscr{P}(A)
\\
y \in x
}& 
\makecell{ %Goal
$  y \in \ps(A)$ 
}
\\ \hline \end{tabular}

\bigskip
By the definition of a power set
\newline
\nopagebreak
\begin{tabular}{|c|c|} \hline
Givens & Goal \\ \hline
\makecell{ %Givens
$A \subseteq  \mathscr{P}(A)$  
\\ 
x \in \mathscr{P}(A)
\\
y \in x
}& 
\makecell{ %Goal
$  y \subseteq A$ 
}
\\ \hline \end{tabular}

\bigskip
By the definition of a subset
\newline
\nopagebreak
\begin{tabular}{|c|c|} \hline
Givens & Goal \\ \hline
\makecell{ %Givens
$A \subseteq  \mathscr{P}(A)$  
\\ 
x \in \mathscr{P}(A)
\\
y \in x
}& 
\makecell{ %Goal
$\forall z ( z \in y \implies z \in A )$ 
}
\\ \hline \end{tabular}


\bigskip
Let z be an arbitrary element of y
\newline
\nopagebreak
\begin{tabular}{|c|c|} \hline
Givens & Goal \\ \hline
\makecell{ %Givens
$A \subseteq  \mathscr{P}(A)$  
\\ 
x \in \mathscr{P}(A)
\\
y \in x
}& 
\makecell{ %Goal
$ z \in y \implies z \in A $ 
}
\\ \hline \end{tabular}

\bigskip
Assume z \in y
\newline
\nopagebreak
\begin{tabular}{|c|c|} \hline
Givens & Goal \\ \hline
\makecell{ %Givens
$A \subseteq  \mathscr{P}(A)$  
\\ 
x \in \mathscr{P}(A)
\\
y \in x
\\
z \in y
}& 
\makecell{ %Goal
$ z \in A $ 
}
\\ \hline \end{tabular}

\bigskip
By the definition of a power set
\newline
\nopagebreak
\begin{tabular}{|c|c|} \hline
Givens & Goal \\ \hline
\makecell{ %Givens
$A \subseteq  \mathscr{P}(A)$  
\\ 
x \subseteq A 
\\
y \in x
\\
z \in y
}& 
\makecell{ %Goal
$ z \in A $ 
}
\\ \hline \end{tabular}
 
\bigskip
By the definition of a subset
\newline
\nopagebreak
\begin{tabular}{|c|c|} \hline
Givens & Goal \\ \hline
\makecell{ %Givens
$A \subseteq  \mathscr{P}(A)$  
\\ 
$\forall k ( k \in x \implies k \in A )$ 
\\
y \in x
\\
z \in y
}& 
\makecell{ %Goal
$ z \in A $ 
}
\\ \hline \end{tabular}
 
\bigskip
Universal instantiation (k = y)
\newline
\nopagebreak
\begin{tabular}{|c|c|} \hline
Givens & Goal \\ \hline
\makecell{ %Givens
$A \subseteq  \mathscr{P}(A)$  
\\ 
$\forall k ( k \in x \implies k \in A )$ 
\\ 
$ y \in x \implies y \in A $ 
\\
y \in x
\\
z \in y
}& 
\makecell{ %Goal
$ z \in A $ 
}
\\ \hline \end{tabular}
 
\bigskip
Modus Ponens (y \in x)
\nopagebreak
\newline
\begin{tabular}{|c|c|} \hline
Givens & Goal \\ \hline
\makecell{ %Givens
$A \subseteq  \mathscr{P}(A)$  
\\ 
$\forall k ( k \in x \implies k \in A )$ 
\\ 
$ y \in A $ 
\\
y \in x
\\
z \in y
}& 
\makecell{ %Goal
$ z \in A $ 
}
\\ \hline \end{tabular}
 
 \bigskip
By the definition of a subset ($A \subseteq  \mathscr{P}(A)$  )
\nopagebreak
\newline
\begin{tabular}{|c|c|} \hline
Givens & Goal \\ \hline
\makecell{ %Givens
$\forall m ( m \in A \implies m \in \ps(A) )$ 
\\ 
$\forall k ( k \in x \implies k \in A )$ 
\\ 
$ y \in A $ 
\\
y \in x
\\
z \in y
}& 
\makecell{ %Goal
$ z \in A $ 
}
\\ \hline \end{tabular}

 \bigskip
Universal instantiation (m = y)
\nopagebreak
\newline
\begin{tabular}{|c|c|} \hline
Givens & Goal \\ \hline
\makecell{ %Givens
$\forall m ( m \in A \implies m \in \ps(A) )$ 
\\
y \in A \implies y \in \ps(A) 
\\ 
$\forall k ( k \in x \implies k \in A )$ 
\\ 
$ y \in A $ 
\\
y \in x
\\
z \in y
}& 
\makecell{ %Goal
$ z \in A $ 
}
\\ \hline \end{tabular}

 \bigskip
Modus Ponens ($ y \in A $ )
\nopagebreak
\newline
\begin{tabular}{|c|c|} \hline
Givens & Goal \\ \hline
\makecell{ %Givens
$\forall k ( k \in x \implies k \in A )$ 
\\
$\forall m ( m \in A \implies m \in \ps(A) )$ 
\\
y \in \ps(A) 
\\ 
$ y \in A $ 
\\
y \in x
\\
z \in y
}& 
\makecell{ %Goal
$ z \in A $ 
}
\\ \hline \end{tabular}

 \bigskip
By the definition of a power set
\nopagebreak
\newline
\begin{tabular}{|c|c|} \hline
Givens & Goal \\ \hline
\makecell{ %Givens
$\forall k ( k \in x \implies k \in A )$ 
\\
$\forall m ( m \in A \implies m \in \ps(A) )$ 
\\
y \subseteq A 
\\ 
$ y \in A $ 
\\
y \in x
\\
z \in y
}& 
\makecell{ %Goal
$ z \in A $ 
}
\\ \hline \end{tabular}


\bigskip
By the definition of a subset
\nopagebreak
\newline
\begin{tabular}{|c|c|} \hline
Givens & Goal \\ \hline
\makecell{ %Givens
$\forall k ( k \in x \implies k \in A )$ 
\\
$\forall m ( m \in A \implies m \in \ps(A) )$ 
\\
$\forall n ( n \in y \implies n \in A )$ 
\\ 
$ y \in A $ 
\\
y \in x
\\
z \in y
}& 
\makecell{ %Goal
$ z \in A $ 
}
\\ \hline \end{tabular}

\bigskip
Universal Instantiation (n = z)
\nopagebreak
\newline
\begin{tabular}{|c|c|} \hline
Givens & Goal \\ \hline
\makecell{ %Givens
$\forall k ( k \in x \implies k \in A )$ 
\\
$\forall m ( m \in A \implies m \in \ps(A) )$ 
\\
$\forall n ( n \in y \implies n \in A )$ 
\\
$ z \in y \implies z \in A $ 
\\ 
$ y \in A $ 
\\
y \in x
\\
z \in y
}& 
\makecell{ %Goal
$ z \in A $ 
}
\\ \hline \end{tabular}

\bigskip
Modus Ponens (z \in y)
\nopagebreak
\newline
\begin{tabular}{|c|c|} \hline
Givens & Goal \\ \hline
\makecell{ %Givens
$\forall k ( k \in x \implies k \in A )$ 
\\
$\forall m ( m \in A \implies m \in \ps(A) )$ 
\\
$\forall n ( n \in y \implies n \in A )$ 
\\
$ z \in A $ 
\\ 
$ y \in A $ 
\\
y \in x
\\
z \in y
}& 
\makecell{ %Goal
$ z \in A $ 
}
\\ \hline \end{tabular}

\bigskip
Now the goal follows from the givens

\section{Final Solution}
\subsection{Task}
Suppose  $A \subseteq  \mathscr{P}(A)$. Prove that $  \mathscr{P}(A) \subseteq \mathscr{P}(\mathscr{P}(A))$
\subsection{Solution}
Assume $A \subseteq  \ps (A)$. Let x be an arbitrary element. Assume $x \in \ps(A)$. Let y be an arbitrary element of x. Assume $y \in x$. Let z be an arbitrary element of y. Assume $z \in y$. Since $x \in \ps (A)$, $x \subseteq A$. Since $y \in x$, $y \in A$. Since $y \in \ps(A)$, $y \subseteq A$. Since $z \in y$, $z \in A$. Since z is an arbitrary element of y and $z \in A$, $y \subseteq A$ and $y \in \ps (A)$. Since y is an arbitrary element of x and $y \in \ps (A)$, $x \subseteq \ps (A)$ and $x \in \ps (\ps(A))$. Since x is an arbitrary element of $\ps (A)$ and $x \in \ps (\ps(A))$, it follows that $\ps (A) \subseteq \ps (\ps(A)). \blacksquare$

\end{document}
